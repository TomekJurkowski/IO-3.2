\RequirePackage{atbegshi}
\documentclass{beamer}

\usepackage [MeX]{polski}
\usepackage [utf8] {inputenc}

\usecolortheme[named=blue]{structure}

\mode<presentation>
{
  \usetheme{Warsaw}
  \setbeamercovered{transparent}
  \setbeamertemplate{items}[ball]
  \setbeamertemplate{theorems}[numbered]
  \setbeamertemplate{footline}[frame number]
}

\usepackage{beamerthemesplit}
\usepackage{graphics}
\usepackage{graphicx}
\usepackage{hyperref}

\usepackage{listings}
  \lstdefinestyle{customc}{
  belowcaptionskip=1\baselineskip,
  breaklines=true,
  frame=L,
  xleftmargin=\parindent,
  language=C,
  showstringspaces=false,
  basicstyle=\footnotesize\ttfamily,
  keywordstyle=\bfseries\color{green!40!black},
  commentstyle=\itshape\color{purple!40!black},
  identifierstyle=\color{blue},
  stringstyle=\color{orange},
}


\usepackage[polish]{babel}
\usepackage[T1]{fontenc}
\usepackage{verbatim}

%\renewcommand*\rmdefault{ppl}
\usepackage{lmodern}

\title
  [Księgowość]
  {Prezentacja systemu księgowego}
\author[grupa 3, podgrupa 2]{Jakub Kuszneruk, Tomasz Jurkowski, Piotr Ostrowski}
\institute[Uniwersytet Warszawski]{
Wydział Matematyki Informatyki i Mechaniki}
\date[2013]{Warszawa, 28.05.2013}
\subject{Computational Sciences}

\pgfdeclaremask{fsu}{fsu_logo}
\pgfdeclareimage[mask=fsu,width=1cm]{fsu-logo}{fsu_logo}

\begin{document}
%-----------------------------------------------------------------------------80
  \begin{frame}{}
    \titlepage
  \end{frame}
%-----------------------------------------------------------------------------80
\section{Wstęp}
%-----------------------------------------------------------------------------80
  \begin{frame}
    \tableofcontents
  \end{frame}
%-----------------------------------------------------------------------------80
\section{e-księgowość}

  \begin{frame}{charakterystyka}
    \begin{block}{}
      Jest to system dedykowany ludziom prowadzącym indywidualną działalność gospodarczą jak i większym firmom, wspierający działanie księgowości.
    \end{block}
  \end{frame}
%-----------------------------------------------------------------------------80
  \begin{frame}{charakterystyka}
    \begin{block}{}
      Głównym zadaniem naszego systemu jest optymalizacja procesów finansowych przedsiębiorstwa, do której zaliczymy ułatiwenie pracy księgowości oraz skrócenie czasu pracy potrzebnego na przeprowadzenie tychże procesów.
    \end{block}
  \end{frame}
%-----------------------------------------------------------------------------80
  \begin{frame}{charakterystyka}
    \begin{block}{}
      Niezwykle ważnym aspektem ,,e-księgowości'' jest fakt, iż jest to aplikacja ,,webowa'' - nie wymagająca instalacji aplikacji na naszym komputerze.
    \end{block}
  \end{frame}
%-----------------------------------------------------------------------------80
\section{Funkcjonalności}

  \begin{frame}
    \begin{block}{}
      Na kolejnych slajdach znajdą się główne funkcjonalności dostarczane przez nasz produkt wraz z ich krótki opisem. Funkcjonalności są przedstawione w kolejności występowania w kolejnych iteracjach.
    \end{block}
  \end{frame}
%-----------------------------------------------------------------------------80
\subsection{1. iteracja}
%-----------------------------------------------------------------------------80
  \begin{frame}
    \begin{block}{Podstawowe zarządzanie użytkownikami}
      Pod tym hasłem kryje się umożliwienie rejestracji nowego użytkownika oraz logowania, bez którego użytkownik nie ma dostępu do pozostałych funkcjonalności systemu.
    \end{block}
  \end{frame}
%-----------------------------------------------------------------------------80
  \begin{frame}
    \begin{block}{Księgowanie faktur sprzedaży}
      Aplikacja daje możliwość zaksięgowania faktury sprzedaży.\\
      Dodatkowo wprowadza automatyczne generowanie numeru faktury\\
      (numer ten wygląda tak: <data wprowadzenia faktury do systemu>/<liczba faktur wprowadzonych tego dnia>)
    \end{block}
  \end{frame}
%-----------------------------------------------------------------------------80
  \begin{frame}
    \begin{block}{Bilans otwarcia}
      Nasza aplikacja umozliwia wprowadzenie uproszczonego bilansu otwarcia, koniecznego do poprawnego funkcjonowania każdej działalności gospodarczej.
    \end{block}
  \end{frame}
%-----------------------------------------------------------------------------80
  \begin{frame}
    \begin{block}{Księga Przychodów i Rozchodów}
      Uproszczona księga Przychodów i Rozchodów zawiera listę zaksięgowanych faktur sprzedaży i zakupu oraz pozycję ,,bilans otwarcia''. Ponadto, widoczne jest podsumowanie przychodów i wydatków (rozchodów) firmy.
    \end{block}
  \end{frame}
%-----------------------------------------------------------------------------80
  \begin{frame}
    \begin{block}{Księga Przychodów i Rozchodów - cdn.}
      Z poziomu ,,księgi PR'' udostępniony jest dostęp do szczegółowego widoku dowolnej faktury.
    \end{block}
  \end{frame}
%-----------------------------------------------------------------------------80
\subsection{2. iteracja}
%-----------------------------------------------------------------------------80
  \begin{frame}
    \begin{block}{Rozliczenie ZUS dla ,,idn''}
      W drugiej iteracji pojawia się możliwość rozliczenia się z ZUS dla przedsiębiorców prowadzących indywidualną działalność gospodarczą.
    \end{block}
  \end{frame}
%-----------------------------------------------------------------------------80
  \begin{frame}
    \begin{block}{VAT7}
      ,,e-księgowość'' daje możliwość wypełnienia deklaracji dla podatku od towarów i usług - VAT7.
    \end{block}
  \end{frame}
%-----------------------------------------------------------------------------80
  \begin{frame}
    \begin{block}{PIT-36L}
      Kolejną funkcjonalnością jest możliwość wypełnienia zeznania o wysokości osiągniętego dochodu (poniesionej straty) w roku podatkowym - PIT-36L.
    \end{block}
  \end{frame}
%-----------------------------------------------------------------------------80
  \begin{frame}
    \begin{block}{Podstawowe wsparcie dla ,,e-Deklaracji''}
      Pojawia się możliwość wysłania formularzy VAT7 oraz PIT-36L poprzez system ,,e-Deklaracje''.
    \end{block}
  \end{frame}
%-----------------------------------------------------------------------------80
  \begin{frame}
    \begin{block}{Ewidencja VAT}
      Kolejną funkcjonalnością wprowadzoną w drugiej iteracji naszego systemu jest możliwość prowadzenia ewidencji VAT.
    \end{block}
  \end{frame}
%-----------------------------------------------------------------------------80
\subsection{3. iteracja}
%-----------------------------------------------------------------------------80
  \begin{frame}
      Funkcjonalności z trzeciej iteracji:
    \begin{itemize}
      \item integracja z kontem bankowym - tzw. ,,pełna księgowość'' 
      \item rozliczenia ZUS dla pracowników
      \item przetrzymywanie informacji o zatrudnionych pracownikach
    \end{itemize}
  \end{frame}
\subsection{4. iteracja}
%-----------------------------------------------------------------------------80
  \begin{frame}
      Funkcjonalności z czwartej iteracji:
    \begin{itemize}
      \item zarządzanie użytkownikami - rozwinięcie 1. iteracji
      \item zarządzanie uprawnieniami użytkowników
      \item system płatności za nasz produkt
    \end{itemize}
  \end{frame}
%-----------------------------------------------------------------------------80
\section{Technologia}
\subsection{Techonologie sieciowe}
%-----------------------------------------------------------------------------80
  \begin{frame}
    \begin{block}{haproxy}
      Nasłuchuj żądania http na porcie 80.
    \end{block}
  \end{frame}
%-----------------------------------------------------------------------------80
  \begin{frame}
    \begin{block}{haproxy}
      Ma zdefiniowany klaster serwerów apache/lighttpd/nginx obsługujących naszą aplikację.
    \end{block}
  \end{frame}
%-----------------------------------------------------------------------------80
\subsection{Technologie web desig}
%-----------------------------------------------------------------------------80
  \begin{frame}
    \begin{block}{Django}
      Podstawowa technologia CMS:
      \begin{itemize}
        \item generowanie html
        \item szybkie tworzenie działającej aplikacji
        \item automatyczna obsługa sesji
      \end{itemize}
    \end{block}
  \end{frame}
%-----------------------------------------------------------------------------80
  \begin{frame}
    \begin{block}{Twitter Bootstrap}
      Zestaw zdefiniowanych klas css i js umożliwiający szybkie tworzenie efektownych stron internetowych.
    \end{block}
  \end{frame}
%-----------------------------------------------------------------------------80
  \begin{frame}
    \begin{block}{jQuery}
      lekka biblioteka programistyczna dla języka JavaScript, ułatwiająca manipulację drzewem DOM:
      \begin{itemize}
        \item modyfikacja elementów drzewa
        \item dodwanie/usuwanie elementów
        \item tworzenie efektownych animacji
      \end{itemize}
    \end{block}
  \end{frame}
%-----------------------------------------------------------------------------80
\subsection{Sklalowalność}
%-----------------------------------------------------------------------------80
  \begin{frame}
    \begin{block}{Frontend:}
      \begin{itemize}
        \item haproxy nasłuchuje na porcie 80, jedyna publicznie dostępna maszyna
        \item opcjonalnie używa varnisha/filtruje ruch
        \item rozbija ruch na N (u nas N=2) maszyn o nazwach app[1..N]
      \end{itemize}
    \end{block}
  \end{frame}
%-----------------------------------------------------------------------------80
  \begin{frame}
    \begin{block}{App[1..N]}
      \begin{itemize}
        \item nasłuchują na wybranym porcie połączeń od haproxy
        \item na każdej z nich chodzi identyczne Django
        \item połączenia DB (SQLlite) idą do kolejnych maszyn DB[1..N]
      \end{itemize}
    \end{block}
  \end{frame}
%-----------------------------------------------------------------------------80
  \begin{frame}
    \begin{block}{DB[1..N]}
      \begin{itemize}
        \item obsługa db
        \item pojedyncza baza (SQLlite)
        \item możliwość replikacji z 2. bazą (loadbalncing pomiędzy bazami)
      \end{itemize}
    \end{block}
  \end{frame}
%-----------------------------------------------------------------------------80
\section{Napotkane Problemy}
\subsection{Problemy koncepcyjne}
%-----------------------------------------------------------------------------80
  \begin{frame}
    \begin{block}{}
      Głównym problemem jaki stanął na naszej drodze to ...
    \end{block}
  \end{frame}
%-----------------------------------------------------------------------------80
  \begin{frame}
    \begin{block}{}
      Głównym problemem jaki stanął na naszej drodze to ...
    \end{block}
    \begin{block}{}
      ... nieznajomość niezwykle obszernej i trudnej dla laika dziedziny jaką jest prowadzenie księgowości.
    \end{block}
  \end{frame}
%-----------------------------------------------------------------------------80
  \begin{frame}
    \begin{block}{}
      Jak można było przypuszczać, praktycznie ,,zerowe'' doświadczenie naszego zespołu w prowadzeniu księgowości przysporzyło nam sporo problemów, które zaczęły się już w najwcześniejszych fazach (tworzenie diagramów m.in. przypadków użycia) i ciągną się do dnia dzisiejszego, czyli do fazy implementacji.
    \end{block}
  \end{frame}
%-----------------------------------------------------------------------------80
  \begin{frame}
    \begin{block}{}
      Co zatem zrobiliśmy z tym problemem? Niestety nie udało nam się znaleźć jakiegoś genialnego tutoriala pt. ,,jak przyswoić wiedzę o księgowości w 24h'', a także nie było nas stać na wynajęcie konsultanta.
    \end{block}
  \end{frame}
%-----------------------------------------------------------------------------80
  \begin{frame}
    \begin{block}{}
      Nie pozostało nam zatem nic innego jak przebijanie się przez wikipedię, analizę podobnych systemów dostępnych na rynku oraz atakowanie pytaniami Grześka:)
    \end{block}
  \end{frame}
%-----------------------------------------------------------------------------80
\subsection{problemy w implementacji}
%-----------------------------------------------------------------------------80
  \begin{frame}
    \begin{block}{}
      Kolejnym problemem technicznym, na który się natknęliśmy to jak umożliwić użytkownikowi wygodne dodawanie faktur zakupu.
    \end{block}
  \end{frame}
%-----------------------------------------------------------------------------80
  \begin{frame}
    \begin{block}{}
      Precyzyjniej, problem sprowadza się do wypełnienia kilkunastu pól formularza faktury (np. nazwa sprzedawcy, data wystawienia itd.), a następnie dodanie za pomocą kolejnego formularza pozycji do tejże faktury, z czego liczba pozycji powinna być większa od zera i jednocześnie nieograniczona z góry.
    \end{block}
  \end{frame}
%-----------------------------------------------------------------------------80
  \begin{frame}
    \begin{block}{}
      W przypadku, gdy chcemy dodać więcej pozycji niż jest na to miejsce w formularzu powinna być możliwość dostawienia kolejnej pozycji.
    \end{block}
  \end{frame}
%-----------------------------------------------------------------------------80
  \begin{frame}
    \begin{block}{}
      Rozwiązanie jakie zastosowaliśmy polega na podzieleniu tworzenia faktury zakupu na dwa etapy: w pierwszym widoku użytkownik dostaje do uzupełnienia wszelkie pola faktury poza pozycjami i następnie może się przenieść do drugiego widoku, umożliwiającego dostawienie pozycji do faktury. W drugim widoku pojawia się klawisz ,,Dodaj kolejną pozycję'' dostawiający pola dla dodatkowej pozycji i klawisz ,,Zaksięguj fakturę'' - bez komentarza.
    \end{block}
  \end{frame}
%-----------------------------------------------------------------------------80
  \begin{frame}
    \begin{block}{}
      Alternatywnym rozwiązaniem wartym rozważenia, ale wymagającym nieco większej znajomości technologii jest zastosowanie mechanizmu formset-ów udostępnianego przez Django. W skrócie umożliwia on dodawanie więcej niż jednego formularza do pojedynczego widoku. Po szczegóły odsyłamy do dokumentacji.
    \end{block}
  \end{frame}
%-----------------------------------------------------------------------------80

\end{document}
